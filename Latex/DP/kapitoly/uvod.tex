\chapter*{\'{U}vod}

\addcontentsline{toc}{chapter}{\'{U}vod}

    Tato pr\'{a}ce se zab\'{y}v\'{a} problematikou simulace proud\v{e}n\'{\i} tekutin (anglicky computational fluid dynamics, d\'{a}le CFD), za pou\v{z}it\'{\i} m\v{r}\'{\i}\v{z}kov\'{e} Boltzmannovy metody (LBM), co\v{z} pat\v{r}\'{\i} mezi nejsklonovan\v{e}j\v{s}\'{\i} t\'{e}mata na poli numerick\'{e} matematiky. Z\'{a}rove\v{n} je CFD velmi hojn\v{e} vyu\v{z}\'{\i}van\'{e} v praxi, nap\v{r}\'{\i}klad ve Formuli 1. V automobilov\'{e}m pr\r{u}myslu obecn\v{e} je v posledn\'{\i}ch letech st\'{a}le v\v{e}t\v{s}\'{\i} d\r{u}raz kladen na numerick\'{e} simulace narozd\'{\i}l od experiment\r{u}, kter\'{e} b\'{y}vaj\'{\i} finan\v{c}n\v{e} mnohem n\'{a}ro\v{c}n\v{e}j\v{s}\'{\i}.

    V p\v{r}edch\'{a}zej\'{\i}c\'{\i} bakal\'{a}\v{r}sk\'{e} pr\'{a}ci \cite{BP_DH} bylo zkoum\'{a}no neizoterm\'{a}ln\'{\i} proud\v{e}n\'{\i} okolo p\v{r}ek\'{a}\v{z}ek a spolu s n\'{\i}m i aerodynamick\'{e} vlastnosti tohoto proud\v{e}n\'{\i}. V t\'{e}to pr\'{a}ci je hlavn\'{\i}m c\'{\i}lem simulovat izoterm\'{a}ln\'{\i} proud\v{e}n\'{\i} spolu s p\v{r}estupem tepla mezi r\r{u}zn\'{y}mi typy prost\v{r}ed\'{\i}.

    % V r\'{a}mci t\'{e}to pr\'{a}ce byla nav\'{a}z\'{a}na spolupr\'{a}ce s t\'{y}mem eForce FEE Prague Formula \cite{eForce_FS}, kter\'{y} p\r{u}sob\'{\i} v studentsk\'{e} sout\v{e}\v{z}i Formula Student \cite{FS} s elektrickou formul\'{\i}. C\'{\i}lem sout\v{e}\v{z}e je postavit z\'{a}vodn\'{\i} auto, ale nen\'{\i} to pouze o v\'{y}konu -- hodnot\'{\i} se tak\'{e} konstrukce, pl\'{a}nov\'{a}n\'{\i} rozpo\v{c}tu i marketingov\'{y} pl\'{a}n. V p\v{r}\'{\i}pad\v{e} pou\v{z}it\'{\i} CFD pro izoterm\'{a}ln\'{\i}ho proud\v{e}n\'{\i} se nab\'{\i}z\'{\i} zkoum\'{a}n\'{\i} chlad\'{\i}c\'{\i}ho okruhu popsan\'{e}ho v t\'{e}to pr\'{a}ci. Tato pr\'{a}ce se zam\v{e}\v{r}uje v\'{y}hradn\v{e} na samotn\'{y} radi\'{a}tor neboli v\'{y}m\v{e}n\'{\i}k tepla.

    Pr\'{a}ce je rozd\v{e}lena do \v{c}ty\v{r} kapitol. V prvn\'{\i} kapitole je p\v{r}edstaven matematick\'{y} model dynamiky tekutin 
    formulac\'{\i} \'{u}lohy. 
    
    Ve druh\'{e} kapitole je \v{c}ten\'{a}\v{r} sezn\'{a}men s m\v{r}\'{\i}\v{z}kovou Boltzmannovou metodou. Pops\'{a}ny jsou z\'{a}kladn\'{\i} principy metody, numerick\'{a} sch\'{e}mata pou\v{z}\'{\i}van\'{a} p\v{r}i simulac\'{\i}ch, spolu s p\v{r}edstaven\'{\i}m okrajov\'{y}ch podm\'{\i}nek a algoritmu LBM.
    V t\'{e}to kapitole je tak\'{e} pops\'{a}na p\v{r}estupov\'{a} okrajov\'{a} podm\'{\i}nka implementovan\'{a} v r\'{a}mci t\'{e}to pr\'{a}ce. 
    
    T\v{r}et\'{\i} kapitola je v\v{e}nov\'{a}na pozn\'{a}mk\'{a}m k implementaci k\'{o}du

    D\'{a}le zde \v{c}ten\'{a}\v{r} nalezne popis datov\'{y}ch struktur pou\v{z}it\'{y}ch k implementaci LBM. V posledn\'{\i} kapitole jsou prezentov\'{a}ny v\'{y}sledky aplikace m\v{r}\'{\i}\v{z}kov\'{e} Boltzmannovy metody na matematick\'{y} model zaveden\'{y} v prvn\'{\i} kapitole. V t\'{e}to \v{c}\'{a}sti je nejprve komentov\'{a}na implementace pole pro r\r{u}zn\'{e} difuzn\'{\i} koeficienty, d\'{a}le je zde ov\v{e}\v{r}ov\'{a}na p\v{r}estupov\'{a} okrajov\'{a} podm\'{\i}nka a hled\'{a}ny optim\'{a}ln\'{\i} rozm\v{e}ry chladi\v{c}e. Na z\'{a}v\v{e}r t\'{e}to kapitoly jsou diskutov\'{a}na mo\v{z}n\'{e} sm\v{e}ry dal\v{s}\'{\i}ho zlep\v{s}en\'{\i} modelu.

    Ke zm\'{\i}n\v{e}n\'{y}m numerick\'{y}m simulac\'{\i}m byl vyu\v{z}\'{\i}v\'{a}n k\'{o}d LBM vyv\'{\i}jen\'{y} ji\v{z} n\v{e}kolik let na KM FJFI \v{C}VUT v Praze. K\'{o}d je naps\'{a}n v jazyce C++ a je v n\v{e}m vyu\v{z}ita architektura CUDA umo\v{z}\v{n}uj\'{\i}c\'{\i} paraleln\'{\i} po\v{c}\'{\i}t\'{a}n\'{\i} na grafick\'{y}ch kart\'{a}ch. D\'{a}le k\'{o}d vyu\v{z}\'{\i}v\'{a} knihovny OpenMPI pro paraleln\'{\i} po\v{c}\'{\i}t\'{a}n\'{\i} na v\'{\i}ce grafick\'{y}ch kart\'{a}ch. V\'{y}po\v{c}ty byly prim\'{a}rn\v{e} uskute\v{c}n\v{e}ny na v\'{y}po\v{c}etn\'{\i}m clusteru HELIOS na KM FJFI \v{C}VUT, konkr\'{e}tn\v{e} na grafick\'{y}ch kart\'{a}ch NVIDIA A100 s 80 GB pam\v{e}t\'{\i}. D\'{\i}ky funkcionalit\v{e} OpenMPI bylo mo\v{z}n\'{e} vyu\v{z}\'{\i}t v\v{s}echny \v{c}ty\v{r}i dostupn\'{e} grafick\'{e} karty najednou. Tento k\'{o}d byl n\'{a}sledn\v{e} modifikov\'{a}n dle pot\v{r}eb zad\'{a}n\'{\i} -- byla do n\v{e}j implementov\'{a}na p\v{r}estupov\'{a} okrajov\'{a} podm\'{\i}nka a d\'{a}le bylo p\v{r}id\'{a}no difuzn\'{\i} pole umo\v{z}\v{n}uj\'{\i}c\'{\i} nastaven\'{\i} odli\v{s}n\'{e}ho difuzn\'{\i}ho koeficientu pro r\r{u}zn\'{e} objekty v simulaci. 

    % Tato práce se zab\'{y}v\'{a} problematikou simulace proud\v{e}n\'{\i} tekutin (z anglick\'{e}ho computational fluid dynamics, d\'{a}le CFD), co\v{z} je jedno z nejv\'{\i}ce zkouman\'{y}ch odv\v{e}tv\'{\i} numerick\'{e} matematiky a z\'{a}rove\v{n} intenzivn\v{e} vyu\v{z}\'{\i}van\'{e} v praxi, nap\v{r}\'{\i}klad ve Formuli 1. Ve zm\'{\i}n\v{e}n\'{e}m motorsportu, ale i automobilov\'{e}m pr\r{u}myslu obecn\v{e}, jsou n\'{a}klady spojen\'{e} s optimalizac\'{\i} voz\r{u} vysok\'{e}. Nejen proto jsou st\'{a}le v\'{\i}ce experiment\'{a}ln\'{\i} modely v ran\'{y}ch f\'{a}z\'{\i}ch projekt\r{u} nahrazov\'{a}ny numerick\'{y}mi simulacemi. Mezi dal\v{s}\'{\i} v\'{y}hody numerick\'{y}ch simulac\'{\i} pat\v{r}\'{\i} mo\v{z}nost snadno je pozm\v{e}nit na jednotliv\'{e} \'{u}lohy.

    % Numerick\'{a} metoda studovan\'{a} v této bakalářské pr\'{a}ci je m\v{r}\'{\i}\v{z}kov\'{a} Boltzmannova metoda (LBM, z~anglick\'{e}ho Lattice-Boltzmann method) vyvinut\'{a} na p\v{r}elomu 80. a 90. let dvac\'{a}t\'{e}ho stolet\'{\i}. D\'{\i}ky mo\v{z}nosti paraleln\'{\i}ch v\'{y}po\v{c}t\r{u} na GPU t\v{e}\v{s}\'{\i} se tato metoda rozmachu hlavn\v{e} v posledn\'{\i}ch patn\'{a}cti letech, kdy jsou na trhu rok od roku v\'{y}konn\v{e}j\v{s}\'{\i} grafick\'{e} karty umo\v{z}\v{n}uj\'{\i}c\'{\i} st\'{a}le n\'{a}ro\v{c}n\v{e}j\v{s}\'{\i} v\'{y}po\v{c}ty. C\'{\i}lem t\'{e}to pr\'{a}ce je kr\'{a}tce nahl\'{e}dnout do problematiky aplikace LBM na simulaci proud\v{e}n\'{\i} tekutiny okolo p\v{r}ek\'{a}\v{z}ky. 

    
    % K numerick\'{y}m simulac\'{\i}m byl vyu\v{z}\'{\i}v\'{a}n k\'{o}d LBM napsan\'{y} v jazyce C++, kter\'{y} je ji\v{z} n\v{e}kolik let vyv\'{\i}jen\'{y} na KM FJFI \v{C}VUT v Praze. K\'{o}d vyu\v{z}\'{\i}v\'{a} softwarov\'{e} architektury CUDA, kter\'{a} umo\v{z}\v{n}uje paraleln\'{\i} v\'{y}po\v{c}y na grafick\'{y}ch kart\'{a}ch. Tento k\'{o}d byl modifikov\'{a}n dle pot\v{r}eb zad\'{a}n\'{\i} pr\'{a}ce, zejm\'{e}na v~n\v{e}m byla implementov\'{a}na metoda v\'{y}m\v{e}ny hybnosti pro v\'{y}po\v{c}et s\'{\i}ly p\r{u}sob\'{\i}c\'{\i} na t\v{e}leso p\v{r}i obt\'{e}k\'{a}n\'{\i}, s \v{c}\'{\i}m\v{z} souvis\'{\i} i v\'{y}po\v{c}et bezrozm\v{e}rn\'{y}ch koeficient\r{u}. Pro implementaci metody v\'{y}m\v{e}ny hybnosti bylo nejprve nutn\'{e} na grafick\'{e} kart\v{e} rozd\v{e}lit v\'{y}po\v{c}et kolize a \v{s}\'{\i}\v{r}en\'{\i}, abychom z\'{\i}skali spr\'{a}vn\'{e} hodnoty k~v\'{y}po\v{c}tu s\'{\i}ly. D\'{a}le se v r\'{a}mci t\'{e}to pr\'{a}ce poda\v{r}ilo roz\v{s}\'{\i}\v{r}it SDL o ukazatel fluktuac\'{\i} v re\'{a}ln\'{e}m \v{c}ase.
    
    % Pr\'{a}ce je rozd\v{e}lena do t\v{r}\'{\i} kapitol. Nejprve se \v{c}ten\'{a}\v{r} sezn\'{a}m\'{\i} s matematick\'{y}m modelem dynamiky tekutin, z\'{a}kladn\'{\i}mi pojmy z aerodynamiky a formulac\'{\i} \'{u}lohy, v dal\v{s}\'{\i} \v{c}\'{a}sti je p\v{r}edstavena m\v{r}\'{\i}\v{z}kov\'{a} Boltzmannova metoda pou\v{z}it\'{a} p\v{r}i numerick\'{y}ch simulac\'{\i}ch. Spolu s p\v{r}edstaven\'{\i}m LBM je v t\'{e}to \v{c}\'{a}sti uveden i algoritmus pro v\'{y}po\v{c}et s\'{\i}ly metodou v\'{y}m\v{e}ny hybnosti, kter\'{y} byl v r\'{a}mci t\'{e}to pr\'{a}ce implementov\'{a}n. Posledn\'{\i}, t\v{r}et\'{\i} \v{c}\'{a}st, se v\v{e}nuje v\'{y}sledk\r{u}m aplikace m\v{r}\'{\i}\v{z}kov\'{e} Boltzmannovy metody. V t\'{e}to \v{c}\'{a}sti je nejprve diskutov\'{a}na spr\'{a}vnost implementace v\'{y}po\v{c}tu s\'{\i}ly ve 2D modelu, n\'{a}sledn\v{e} je tento v\'{y}po\v{c}et diskutov\'{a}n i~ve 3D. K ov\v{e}\v{r}en\'{\i} se vyu\v{z}ije referen\v{c}n\'{\i}ch hodnot publikovan\'{y}ch v \cite{schafer1996benchmark}.  }Prvn\'{\i} kapitola byla v\v{e}nov\'{a}na zasv\v{e}cen\'{\i} \v{c}ten\'{a}\v{r}e do matematick\'{e}ho modelu, kter\'{y} se nach\'{a}zel na pozad\'{\i} modelovan\'{e}ho jevu. V r\'{a}mci t\'{e}to kapitoly byly uvedeny rovnice popisuj\'{\i}c\'{\i} dynamiku tekutiny spolu s popisem chlad\'{\i}c\'{\i}ho syst\'{e}mu studentsk\'{e} elektrick\'{e} formule. Druh\'{a} kapitola se zam\v{e}\v{r}ovala na samotnou numerickou metodu pou\v{z}itou k simulac\'{\i} - m\v{r}\'{\i}\v{z}kovou Boltzmannovu metodu (LBM). V t\'{e}to \v{c}\'{a}sti byla pops\'{a}na p\v{r}estupov\'{a} okrajov\'{a} podm\'{\i}nka, kter\'{a} byla v r\'{a}mci pr\'{a}ce implementov\'{a}na.